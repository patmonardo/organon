BOOK ONE

The Doctrine of Being

WITH WHAT MUST THE BEGINNING OF SCIENCE BE MADE?

It is only in recent times that
there has been a new awareness of
the difficulty of finding a beginning in philosophy,
and the reason for this difficulty,
and so also the possibility of resolving it,
have been discussed in a variety of ways.
The beginning of philosophy must be either
something mediated or something immediate,
and it is easy to show that it can be
neither the one nor the other;
so either way of beginning runs into contradiction.

The principle of a philosophy also
expresses a beginning, of course,
but not so much a subjective as an objective one,
the beginning of all things.
The principle is a somehow determinate content:
“water,” “the one,” “nous,” “idea,” or “substance,” “monad,” etc.
or, if it designates the nature of cognition
and is therefore meant simply as a criterion
rather than an objective determination,
as “thinking,” “intuition,” “sensation,” “I,” even “subjectivity,”
then here too the interest still lies in the content determination.
The beginning as such, on the other hand,
as something subjective in the sense that
it is an accidental way of introducing the exposition,
is left unconsidered, a matter of indifference,
and consequently also the need to ask
with what a beginning should be made remains
of no importance in face of the need for the principle
in which alone the interest of the fact seems to lie,
the interest as to what is the truth,
the absolute ground of everything.

But the modern perplexity about a beginning
proceeds from a further need which escapes
those who are either busy demonstrating
their principle dogmatically
or skeptically looking for a subjective criterion
against dogmatic philosophizing,
and is outright denied by those who begin,
like a shot from a pistol,
from their inner revelation,
from faith, intellectual intuition, etc.
and who would be exempt from method and logic.
If earlier abstract thought is at first interested only
in the principle as content,
but is driven as philosophical culture advances
to the other side to pay attention to
the conduct of the cognitive process,
then the subjective activity has also been grasped
as an essential moment of objective truth,
and with this there comes the need to unite
the method with the content,
the form with the principle.
Thus the principle ought to be
also the beginning,
and that which has priority for thinking
ought to be also the first
in the process of thinking.

Here we only have to consider
how the logical beginning appears.
The two sides from which it can be
taken have already been named,
namely either by way of mediation as result,
or immediately as beginning proper.
This is not the place to discuss the question
apparently so important to present-day culture,
whether the knowledge of truth is
an immediate awareness that begins absolutely, a faith,
or rather a mediated knowledge.
In so far as the issue allows passing treatment,
this has already been done elsewhere
(in my Encyclopedia of the Philosophical Sciences, 3rd edn,
in the Prefatory Concept, §§21ff.).
Here we may quote from it only this,
that there is nothing in heaven or nature or spirit or anywhere else
that does not contain just as much immediacy as mediation,
so that both these determinations prove
to be unseparated and inseparable
and the opposition between them nothing real.
As for a scientific discussion,
a case in point is every logical proposition
in which we find the determinations of immediacy and mediacy
and where there is also entailed, therefore,
a discussion of their opposition and their truth.
This opposition, when connected
to thinking, to knowledge, to cognition,
assumes the more concrete shape
of immediate or mediated knowledge,
and it is then up to the science of logic
to consider the nature of cognition in general,
while the more concrete forms of
the same cognition fall within the scope of
the science of spirit and the phenomenology of spirit.
But to want to clarify the nature of cognition prior to science
is to demand that it should be discussed outside science,
and outside science this cannot be done,
at least not in the scientific manner
which alone is the issue here.

A beginning is logical in that it is to be made in
the element of a free, self-contained thought, in pure knowledge;
it is thereby mediated, for pure knowledge is
the ultimate and absolute truth of consciousness.
We said in the Introduction that the Phenomenology of Spirit is
the science of consciousness, its exposition;
that consciousness has the concept of science,
that is, pure knowledge, for its result.
To this extent, logic has for its presupposition
the science of spirit in its appearance,
a science which contains the necessity,
and therefore demonstrates the truth,
of the stand-point which is pure knowledge
and of its mediation.
In this science of spirit in its appearance
the beginning is made from empirical, sensuous consciousness,
and it is this consciousness which is
immediate knowledge in the strict sense;
there, in this science, is where its nature is discussed.
Any other consciousness, such as faith in divine truths,
inner experience, knowledge through inner revelation, etc.,
proves upon cursory reflection to be very ill-suited
as an instance of immediate knowledge.
In the said treatise, immediate consciousness is also
that which in the science comes first and immediately
and is therefore a presupposition;
but in logic the presupposition is
what has proved itself to be the result
of that preceding consideration,
namely the idea as pure knowledge.
Logic is the pure science, that is,
pure knowledge in the full compass of its development.
But in that result the idea has the determination of
a certainty that has become truth;
it is a certainty which, on the one hand, no longer stands
over and against a subject matter confronting it externally
but has interiorized it, is knowingly aware
that the subject matter is itself;
and, on the other hand, has relinquished
any knowledge of itself that would oppose it to objectivity
and would reduce the latter to a nothing;
it has externalized this subjectivity
and is at one with its externalization.

Now starting with this determination of pure knowledge,
all that we have to do to ensure that the beginning will remain
immanent to the science of this knowledge is to consider,
or rather, setting aside every reflection,
simply to take up, what is there before us.
Pure knowledge, thus withdrawn into this unity,
has sublated every reference to an other and to mediation;
it is without distinctions and as thus distinctionless
it ceases to be knowledge;
what we have before us is only simple immediacy.

Simple immediacy is itself an expression of reflection;
it refers to the distinction from what is mediated.
The true expression of this simple immediacy is therefore pure being.
Just as pure knowledge should mean nothing but knowledge as such,
so also pure being should mean nothing but being in general;
being, and nothing else, without further determination and filling.

Being is what makes the beginning here;
it is presented indeed as originating through mediation,
but a mediation which at the same time sublates itself,
and the presupposition is of a pure knowledge
which is the result of finite knowledge, of consciousness.
But if no presupposition is to be made,
if the beginning is itself to be taken immediately,
then the only determination of this beginning is
that it is to be the beginning of logic, of thought as such.
There is only present the resolve,
which can also be viewed as arbitrary,
of considering thinking as such.
The beginning must then be absolute
or, what means the same here, must be an abstract beginning;
and so there is nothing that it may presuppose,
must not be mediated by anything or have a ground,
ought to be rather itself the ground of the entire science.
It must therefore be simply an immediacy,
or rather only immediacy itself.
Just as it cannot have any determination with respect to an other,
so too it cannot have any within;
it cannot have any content, for any content would entail distinction
and the reference of distinct moments to each other,
and hence a mediation.
The beginning is therefore pure being.

After this simple exposition of what alone first belongs to
this simplest of all simples, the logical beginning,
we may add the following further reflections which should not serve,
however, as elucidation and confirmation of the exposition
[this is complete by itself]
but are rather occasioned by notions and reflections
which may come our way beforehand and yet,
like all other prejudices that antedate the science of logic,
must be disposed of within the science itself
and are therefore to be patiently deferred until then.

The insight that absolute truth must be a result,
and conversely, that a result presupposes a first truth
which, because it is first, objectively considered is
not necessary and from the subjective side is not known;
this insight has recently given rise to the thought
that philosophy can begin only with something
which is hypothetically and problematically true,
and that at first, therefore,
philosophizing can be only a quest.
This is a view that Reinhold has repeatedly urged
in the later stages of his philosophizing,
and which must be given credit for being
motivated by a genuine interest in
the speculative nature of philosophical beginning.
A critical examination of this view will also be
an occasion for introducing a preliminary understanding of
what progression in logic generally means,
for the view has direct implications
for the nature of this advance.
Indeed, as portrayed by it,
progression in philosophy would be
rather a retrogression and a grounding,
only by virtue of which it then follows as result that
that, with which the beginning was made,
was not just an arbitrary assumption
but was in fact the truth,
and the first truth at that.

It must be admitted that it is an essential consideration
(one which will be found elaborated again within the logic itself)
that progression is a retreat to the ground,
to the origin and the truth on which
that with which the beginning was made,
and from which it is in fact produced, depends.
Thus consciousness, on its forward path from
the immediacy with which it began,
is led back to the absolute knowledge
which is its innermost truth.
This truth, the ground, is then also
that from which the original first proceeds,
the same first which at the beginning
came on the scene as something immediate.
It is most of all in this way that absolute spirit
(which is revealed as the concrete and supreme truth of all being)
comes to be known, as at the end of the development
it freely externalizes itself,
letting itself go into the shape of an immediate being,
resolving itself into the creation of a world
which contains all that fell within the development
preceding that result and which,
through this reversal of position with its beginning,
is converted into something dependent
on the result as principle.
Essential to science is not so much that
a pure immediacy should be the beginning,
but that the whole of science is in itself a circle
in which the first becomes also the last,
and the last also the first.

Conversely, it follows that it is
just as necessary to consider as result
that into which the movement returns as to its ground.
In this respect, the first is just as much the ground,
and the last a derivative;
since the movement makes its start from the first
and by correct inferences arrives at the last
as the ground, this last is result.
Further, the advance from that which constitutes the beginning is
to be considered only as one more determination of the same advance,
so that this beginning remains as the underlying ground of
all that follows without vanishing from it.
The advance does not consist in the derivation of an other,
or in the transition to a truly other:
inasmuch as there is a transition,it is equally sublated again.
Thus the beginning of philosophy is the ever present
and self-preserving foundation of all subsequent developments,
remaining everywhere immanent in its further determinations.

In this advance the beginning thus loses the one-sidedness
that it has when determined simply as something immediate and abstract;
it becomes mediated, and the line of scientific forward movement
consequently turns into a circle.
It also follows that what constitutes the beginning,
because it is something still undeveloped and empty of content,
is not yet truly known at that beginning,
and that only science, and science fully developed,
is the completed cognition of it,
replete with content and finally truly grounded.

But for this reason,
because it is as absolute ground
that the result finally emerges,
the progression of this cognition
is not anything provisory,
still problematic and hypothetical,
but must be determined through the nature
of the matter at issue and of the content itself.
Nor is the said beginning an
arbitrary and only temporary assumption,
or something which seems to be an
arbitrary and tentative presupposition
but of which it is subsequently shown
that to make it the starting point
was indeed the right thing to do;
this is not as when we are instructed
to make certain constructions in order
to aid the proof of a geometrical theorem,
and only in retrospect, in the course of the proof,
does it become apparent that we did well to draw
precisely these lines and then, in the proof itself,
to begin by comparing them or the enclosed angles,
though the line-drawing or the comparing
themselves escape conceptual comprehension.

So we have just given, right within science itself,
the reason why in pure science
the beginning is made with pure being.
This pure being is the unity
into which pure knowledge returns,
or if this knowledge, as form,
is itself still to be kept
distinct from its unity,
then pure being is also its content.
It is in this respect that this pure being,
this absolute immediate,
is just as absolutely mediated.
However, just because it is here as the beginning,
it is just as essential that it should be taken
in the one-sidedness of being purely immediate.
If it were not this pure indeterminacy,
if it were determined,
it would be taken as something mediated,
would already be carried further than itself:
a determinate something has the character of an other
with respect to a first.
It thus lies in the nature of a beginning itself
that it should be being and nothing else.
There is no need, therefore,
of other preparations to enter philosophy,
no need of further reflections or access points.

Nor can we derive a more specific determination
or a more positive content
for the beginning of philosophy
from the fact that it is such a beginning.
For here, at the beginning, where the fact
itself is not yet at hand,
philosophy is an empty word,
a received and yet unjustified notion.
Pure knowledge yields only this negative determination,
namely that the beginning ought to be abstract.
If pure being is taken as the content of pure knowledge,
then the latter must step back from its content,
allowing it free play and without determining it further.
Or again, inasmuch as pure being is to be considered
as the unity into which knowledge has collapsed
when at the highest point of union with its objectification,
knowledge has then disappeared into this unity,
leaving behind no distinction from it
and hence no determination for it.
Nor is there anything else present, any content whatever,
that could be used to make a more determinate beginning with it.

But, it may be said, the determination of being
assumed so far as the beginning can also be let go,
so that the only requirement would be
that a pure beginning should be made.
Nothing would then be at hand
except the beginning itself,
and we must see what this would be.
This position could be suggested also for the benefit
of those who are either not comfortable,
for whatever reason, with beginning with being
and even less with the transition into nothing
that follows from being,
or who simply do not know how else to make a beginning
in a science except by presupposing a representation
which is subsequently analyzed,
the result of the analysis then yielding
the first determinate concept in the science.
If we also want to test this strategy,
we must relinquish every particular object
that we may intend, since the beginning,
as the beginning of thought,
is meant to be entirely abstract,
entirely general, all form with no content;
we must have nothing, therefore,
except the representation of a mere beginning as such.
We have, therefore, only to see
what there is in this representation.

As yet there is nothing, and something is supposed to become.
The beginning is not pure nothing but a nothing, rather,
from which something is to proceed;
also being, therefore, is already contained in the beginning.
Therefore, the beginning contains both, being and nothing;
it is the unity of being and nothing,
or is non-being which is at the same time being,
and being which is at the same time non-being.

Further, being and nothing are present
in the beginning as distinguished;
for the beginning points to something other;
it is a non-being which refers to an other;
that which begins, as yet is not;
it only reaches out to being.
The being contained in the beginning is such, therefore,
that it distances itself from non-being
or sublates it as something which is opposed to it.
But further, that which begins already is,
but is also just as much not yet.
The opposites, being and non-being, are
therefore in immediate union in it;
or the beginning is their undifferentiated unity.

An analysis of the beginning would thus yield
the concept of the unity of being and non-being;
or, in a more reflected form, the concept of the unity
of differentiated and undifferentiated being;
or of the identity of identity and non-identity.
This concept could be regarded as the first,
purest, that is, most abstract, definition of the absolute;
as it would indeed be if the issue were just
the form of definitions and the name of the absolute.
In this sense, just as such an abstract concept
would be the first definition of the absolute,
so all further determinations and developments would be only
more determinate and richer definitions of it.
But let those who are not satisfied with being as the beginning,
since being passes over into nothing
and what emerges is the unity of the two,
let them consider what is more likely to satisfy them:
this beginning that begins with the representation
of the beginning and an analysis of it
(an analysis that is indeed correct yet
equally leads to the unity of being and non-being)
or a beginning which makes being the beginning.

But, regarding this strategy,
there is still a further observation to be made.
The said analysis presupposes that
the representation of the beginning is known;
its strategy follows the example of other sciences.
These presuppose their object and presume that
everyone has the same representation of it
and will find in it roughly the same determinations
which they have collected here or there,
through analysis, comparison, and sundry argumentation,
and they then offer as its representations.
But that which constitutes the absolute beginning must
likewise be something otherwise known;
now, if it is something concrete
and hence in itself variously determined,
then this connectedness which it is
in itself is presupposed as a known;
the connectedness is thereby adduced
as something immediate, which however it is not;
for it is connectedness only as
a connection of distinct elements
and therefore contains mediation within itself.
Further, the accidentality
and the arbitrariness of the analysis
and the specific mode of determination
affect the concrete internally.
Which determinations are elicited depends
on what each individual happens to discover
in his immediate accidental representation.
The connection contained within a concrete something,
within a synthetic unity, is necessary only in so far
as it is not found already given but is produced rather
by the spontaneous return of the moments back into this unity,
a movement which is the opposite of the analytical
procedure that occurs rather within the subject
and is external to the fact itself.

Here we then have the precise reason why
that with which the beginning is to be made
cannot be anything concrete,
anything containing a connection within its self.
It is because, as such, it would presuppose
within itself a process of mediation
and the transition from a first to an other,
of which process the concrete something,
now become a simple, would be the result.
But the beginning ought not itself to be
already a first and an other,
for anything which is
in itself a first and an other
implies that an advance has already been made.
Consequently, that which constitutes
the beginning, the beginning itself,
is to be taken as something unanalyzable,
taken in its simple, unfilled immediacy;
and therefore as being,
as complete emptiness.

If, impatient with this talk of an abstract beginning,
one should say that the beginning is to be made,
not with the beginning, but directly with the fact itself,
well then, this subject matter is nothing else
than that empty being.
For what this subject matter is,
that is precisely what ought to result
only in the course of the science,
what the latter cannot presuppose to know in advance.

On any other form otherwise assumed
in an effort to have a beginning
other than empty being,
that beginning would still
suffer from the same defects.
Let those who are still dissatisfied
with this beginning take upon themselves
the challenge of beginning in some other way
and yet avoiding such defects.

But we cannot leave entirely unmentioned
a more original beginning to philosophy
which has recently gained notoriety,
the beginning with the “I.”
It derived from both the reflection that
all that follows from the first truth
must be deduced from it,
and the need that this first truth
should be something with which one is already acquainted,
and even more than just acquainted,
something of which one is immediately certain.
This proposed beginning is not,
as such, an accidental representation,
or one which might be one thing
to one subject and something else to another.
For the “I,” this immediate consciousness of the self,
appears from the start to be both itself
an immediate something and something with which
we are acquainted in a much deeper sense
than with any other representation;
true, anything else known belongs to this “I,”
but it belongs to it as a content which remains
distinct from it and is therefore accidental;
the “I,” by contrast, is the simple certainty of its self.
But the “I” is, as such, at the same time also a concrete,
or rather, the “I” is the most concrete of all things,
the consciousness of itself as an infinitely manifold world.
Before the “I” can be the beginning and foundation of philosophy,
this concreteness must be excised,
and this is the absolute act by virtue of which
the “I” purifies itself and makes its entrance
into consciousness as abstract “I.”
But this pure “I” is now not immediate,
is not the familiar, ordinary “I” of our consciousness
to which everyone immediately links science.
Truly, that act of excision would be
none other than the elevation to the standpoint of pure knowledge in
which the distinction between subject and object has disappeared.
But as thus immediately demanded,
this elevation is a subjective postulate;
before it proves itself as a valid demand,
the progression of the concrete “I”
from immediate consciousness to pure knowledge
must be demonstratively exhibited within the “I” itself,
through its own necessity.
Without this objective movement, pure knowledge,
also when defined as intellectual intuition,
appears as an arbitrary standpoint,
itself one of those empirical states of consciousness
for which everything depends on whether someone,
though not necessarily somebody else,
discovers it within himself
or is able to produce it there.
But inasmuch as this pure “I” must be essential, pure knowledge;
and pure knowledge is however one which is only posited in
individual consciousness through an absolute act of self-elevation,
is not present in it immediately;
we lose the very advantage which was to derive
from this beginning of philosophy,
namely that it is something with which
everyone is well acquainted,
something which everyone finds within himself
and to which he can attach further reflection;
that pure “I,” on the contrary,
in its abstract, essential nature,
is to ordinary consciousness an unknown,
something that the latter does not find within itself.
What comes with it is rather the disadvantage of
the illusion that we are speaking of something supposedly very familiar,
the “I” of empirical self-consciousness,
whereas at issue is in fact something far removed from the latter.
Determining pure knowledge as “I” acts as
a continuing reminder of the subjective “I”
whose limitations should rather be forgotten;
it leads to the belief that the propositions and relations
which result from the further development
of the “I” occur within ordinary consciousness
and can be found pregiven there,
indeed that the whole issue is about this consciousness.
This mistake, far from bringing clarity,
produces instead an even more glaring and bewildering confusion;
among the public at large, it has occasioned
the crudest of misunderstandings.

Further, as regards the subjective determinateness of the “I” in general,
pure knowledge does remove from it the restriction that it has
when understood as standing in unsurmountable opposition to an object.
But for this reason it would be at least superfluous
still to hold on to this subjective attitude
by determining pure knowledge as “I.”
For this determination not only carries with it
that troublesome duality of subject and object;
on closer examination, it also remains a subjective “I.”
The actual development of the science that proceeds from the “I”
shows that in the course of it the object has and retains
the self-perpetuating determination of an other
with respect to the “I”;
that therefore the “I” from which the start was made
does not have the pure knowledge that has truly overcome
the opposition of consciousness,
but is rather still entangled in appearance.
In this connection, there is the further
essential observation to be made that,
although the “I” might well be determined to be
in itself pure knowledge or intellectual intuition
and declared to be the beginning,
in science we are not concerned with
what is present in itself or as something inner,
but with the external existence rather of what in thought is inner
and with the determinateness which this inner assumes in that existence.
But whatever externalization there might be of
intellectual intuition at the beginning of science,
or if the subject matter of science is called
the eternal, the divine, the absolute,
of the eternal or absolute,
this cannot be anything else than a first, immediate, simple determination.
Whatever richer name be given to it than is expressed by mere being,
the only legitimate consideration is how
such an absolute enters into discursive knowledge
and the enunciation of this knowledge.
Intellectual intuition might well be the violent rejection
of mediation and of demonstrative, external reflection.
However, anything which it says over and above
simple immediacy would be something concrete,
and this concrete would contain a diversity of determinations in it.
But, as already remarked, the enunciation and exposition
of this concrete something is a process of mediation
which starts with one of the determinations and proceeds to another,
even though this other returns to the first;
and this is a movement which, moreover, is
not allowed to be arbitrary or assertoric.
Consequently, that from which the beginning is made in any
such exposition is not something itself concrete
but only the simple immediacy from which the movement proceeds.
Besides, what is lacking if we make something concrete the beginning
is the demonstration which the combination of the determinations
contained in it requires.

Therefore, if in the expression of the absolute,
or the eternal, or God
(and God would have the perfectly undisputed right
that the beginning be made with him),
if in the intuition or the thought of them,
there is more than there is in pure being,
then this more should first emerge in a knowledge
which is discursive and not figurative;
as rich as what is implicitly contained in knowledge may be,
the determination that first emerges in it is something simple,
for it is only in the immediate that no advance is
yet made from one thing to an other.
Consequently, whatever in the richer
representations of the absolute or God
might be said or implied over and above being,
all this is at the beginning
only an empty word and only being;
this simple determination which has no further meaning
besides, this empty something, is as such, therefore,
the beginning of philosophy.

This insight is itself so simple
that this beginning is as beginning
in no need of any preparation or further introduction,
and the only possible purpose of
this preliminary disquisition regarding it
was not to lead up to it
but to dispense rather with all preliminaries.

GENERAL DIVISION OF BEING

Being is determined, first, as against another in general;
secondly, it is internally self-determining;
thirdly, as this preliminary division is cast off,
it is the abstract indeterminateness and immediacy
in which it must be the beginning.

According to the first determination,
being partitions itself off from essence,
for further on in its development it proves to be
in its totality only one sphere of the concept,
and to this sphere as moment it opposes another sphere.
According to the second, it is the sphere
within which fall the determinations
and the entire movement of its reflection.

In this, being will posit itself in three determinations:

I. as determinateness; as such, quality;

II. as sublated determinateness; magnitude, quantity;

III. as qualitatively determined quantity; measure.

This division, as was generally remarked of
such divisions in the Introduction,
is here a preliminary statement;
its determinations must first arise
from the movement of being itself,
and receive their definitions and justification by virtue of it.
As regards the divergence of this division from
the usual listing of the categories,
namely quantity, quality, relation and modality,
(for Kant, incidentally, these are supposed to be
only classifications of his categories,
but are in fact themselves categories,
only more abstract ones;
about this, there is nothing to remark here,
since the entire listing will diverge from
the usual ordering and meaning of
the categories at every point.)

This only can perhaps be remarked,
that the determination of quantity
is ordinarily listed ahead of quality and as a rule
this is done for no given reason.
It has already been shown that
the beginning is made with being as such,
and hence with qualitative being.
It is clear from a comparison of quality with quantity
that the former is by nature first.
For quantity is quality which has already become negative;
magnitude is the determinateness which,
no longer one with being but already distinguished from it,
is the sublated quality that has become indifferent.
It includes the alterability of being
without altering the fact itself,
namely being, of which it is the determination;
qualitative determinateness is on the contrary one with its being,
it neither transcends it nor stays within it
but is its immediate restrictedness.
Hence quality, as the determinateness which is immediate,
is the first and it is with it that the beginning is to be made.

Measure is a relation, not relation in general
but specifically of quality and quantity to each other;
the categories dealt with by Kant under relation
will come up elsewhere in their proper place.
Measure, if one so wishes, can be considered also a modality;
but since with Kant modality is no longer
supposed to make up a determination of content,
but only concerns the reference of the content
to thought, to the subjective,
the result is a totally heterogeneous reference
that does not belong here.

The third determination of being falls within
the section Quality inasmuch as being, as abstract immediacy,
reduces itself to one single determinateness
as against its other determinacies inside its sphere.

SECTION I

Determinateness (Quality)

Being is the indeterminate immediate;
it is free of determinateness with respect to essence,
just as it is still free of any determinateness
that it can receive within itself.
This reflectionless being is being
as it immediately is only within.

Since it is immediate, it is being without quality;
but the character of indeterminateness attaches to it in itself
only in opposition to what is determinate or qualitative.
Determinate being thus comes to stand over and against being in general;
with that, however, the very indeterminateness of being
constitutes its quality.
It will therefore be shown that the first being is
in itself determinate, and therefore secondly,
that it passes over into existence, is existence;
that this latter, however, as finite being, sublates itself
and passes over into the infinite reference of being to itself;
it passes over, thirdly, into being-for-itself.

CHAPTER 1

Being

A. BEING

Being, pure being, without further determination.
In its indeterminate immediacy it is equal only to itself
and also not unequal with respect to another;
it has no difference within it, nor any outwardly.
If any determination or content were posited in it as distinct,
or if it were posited by this determination or content
as distinct from an other,
it would thereby fail to hold fast to its purity.
It is pure indeterminateness and emptiness.
There is nothing to be intuited in it,
if one can speak here of intuiting;
or, it is only this pure empty intuiting itself.
Just as little is anything to be thought in it,
or, it is equally only this empty thinking.
Being, the indeterminate immediate is in fact nothing,
and neither more nor less than nothing.

B. NOTHING

Nothing, pure nothingness;
it is simple equality with itself,
complete emptiness,
complete absence of determination and content;
lack of all distinction within.
In so far as mention can be made here of
intuiting and thinking,
it makes a difference whether something or nothing is
being intuited or thought.
To intuit or to think nothing has therefore a meaning;
the two are distinguished and so nothing is (concretely exists)
in our intuiting or thinking;
or rather it is the empty intuiting and thinking itself,
like pure being.
Nothing is therefore the same determination
or rather absence of determination,
and thus altogether the same as what pure being is.

C. BECOMING

1. Unity of being and nothing

Pure being and pure nothing are therefore the same.
The truth is neither being nor nothing,
but rather that being has passed over into nothing
and nothing into being;
“has passed over,” not passes over.

But the truth is just as much that
they are not without distinction;
it is rather that they are not the same,
that they are absolutely distinct
yet equally unseparated and inseparable,
and that each immediately vanishes in its opposite.

Their truth is therefore this movement of
the immediate vanishing of the one into the other:
becoming, a movement in which the two are distinguished,
but by a distinction which has just as immediately dissolved itself.

2. The moments of becoming

Becoming is the unseparatedness of being and nothing,
not the unity that abstracts from being and nothing;
as the unity of being and nothing
it is rather this determinate unity,
or one in which being and nothing equally are.
However, inasmuch as being and nothing are
each unseparated from its other, each is not.
In this unity, therefore, they are,
but as vanishing, only as sublated.
They sink from their initially represented self-subsistence
into moments which are still distinguished
but at the same time sublated.

Grasped as thus distinguished,
each is in their distinguishedness
a unity with the other.
Becoming thus contains being and nothing as two such unities,
each of which is itself unity of being and nothing;
the one is being as immediate and as reference to nothing;
the other is nothing as immediate and as reference to being;
in these unities the determinations are of unequal value.

Becoming is in this way doubly determined.
In one determination, nothing is the immediate,
that is, the determination begins with nothing
and this refers to being;
that is to say, it passes over into it.
In the other determination, being is the immediate,
that is, the determination begins with being
and this passes over into nothing:
coming-to-be and ceasing-to-be.

Both are the same, becoming,
and even as directions that are so different
they interpenetrate and paralyze each other.
The one is ceasing-to-be;
being passes over into nothing,
but nothing is just as much the opposite of itself,
the passing-over into being, coming-to-be.
This coming-to-be is the other direction;
nothing goes over into being,
but being equally sublates itself
and is rather the passing-over into nothing;
it is ceasing-to-be.
They do not sublate themselves reciprocally
[the one sublating the other externally]
but each rather sublates itself in itself
and is within it the opposite of itself.

3. Sublation of becoming

The equilibrium in which coming-to-be and ceasing-to-be are poised
is in the first place becoming itself.
But this becoming equally collects itself in quiescent unity.
Being and nothing are in it only as vanishing;
becoming itself, however, is only by virtue of their being distinguished.
Their vanishing is therefore the vanishing of becoming,
or the vanishing of the vanishing itself.
Becoming is a ceaseless unrest that collapses into a quiescent result.

This can also be expressed thus:
becoming is the vanishing of being into nothing,
and of nothing into being,
and the vanishing of being and nothing in general;
but at the same time it rests on their being distinct.
It therefore contradicts itself in itself,
because what it unites within itself is self-opposed;
but such a union destroys itself.

This result is a vanishedness, but it is not nothing;
as such, it would be only a relapse into one of
the already sublated determinations
and not the result of nothing and of being.
It is the unity of being and nothing
that has become quiescent simplicity.
But this quiescent simplicity is being,
yet no longer for itself but as determination of the whole.

Becoming, as transition into
the unity of being and nothing,
a unity which is as existent
or has the shape of the one-sided
immediate unity of these moments,
is existence.

CHAPTER 2

Existence

Existence is determinate being;
its determinateness is existent determinateness, quality.
Through its quality, something is opposed to an other;
it is alterable and finite,
negatively determined not only towards an other,
but absolutely within it.
This negation in it,
in contrast at first
with the finite something,
is the infinite;
the abstract opposition
in which these determinations appear
resolves itself into oppositionless infinity,
into being-for-itself.

The treatment of existence is therefore in three divisions:

A. existence as such
B. something and other, finitude
C. qualitative infinity.

Transition

Ideality can be called the
quality of the infinite;
but it is essentially
the process of becoming,
and hence a transition,
like the transition
of becoming into existence.
We must now explicate this transition.
This immanent turning back,
as the sublating of finitude,
of finitude as such
and equally of the negative finitude
that only stands opposite to it,
is only negative finitude,
is self-reference, being.
Since there is negation
in this being,
the latter is existence;
but, further, since the
negation is essentially
negation of the negation,
self-referring negation,
it is the existence that
carries the name of
being-for-itself.

CHAPTER 3

Being-for-itself

In being-for-itself,
qualitative being is brought to completion;
it is infinite being;
the being of the beginning is void of determination;
existence is sublated but only immediately sublated being;
it thus contains, to begin with,
only the first negation, itself immediate;
being is of course retained as well,
and the two are united in existence in simple unity;
for this reason, however,
each is in itself still unlike the other,
and their unity is still not posited.
Existence is therefore the sphere of differentiation,
of dualism, the domain of finitude.
Determinateness is determinateness as such;
being which is relatively, not absolutely, determined.
In being-for-itself, the distinction
between being and determinateness,
or negation, is posited and equalized.
Quality, otherness, limit, as well as reality,
in-itselfness, ought, and so forth, are the
incomplete configurations of negation in being
which are still based on the differentiation of the two.
But since in finitude negation has passed over into infinity,
in the posited negation of negation,
negation is simple self-reference
and in it, therefore, the equalization with being:
absolutely determinate being.

First, being-for-itself is immediately
an existent-for-itself, the one.

Second, the one passes over
into a multiplicity of ones,
repulsion or the otherness of the one
which sublates itself into its ideality, attraction.

Third, we have the alternating
determination of repulsion and attraction
in which the two sink into a state of equilibrium;
and quality, driven to a head in being-for-itself,
passes over into quantity.

A. BEING-FOR-ITSELF AS SUCH

The general concept of being-for-itself has come to light.
The justification for using the expression “being-for-itself”
for that concept would depend on showing that the representation
associated with the expression corresponds to the concept.
So indeed it appears to do.
We say that something is for itself
inasmuch as it sublates otherness,
sublates its connection and community with other,
has rejected them by abstracting from them.
The other is in it only as something sublated, as its moment;
being-for-itself consists in
having thus transcended limitation, its otherness;
it consists in being, as this negation,
the infinite turning back into itself.
In representing to itself an intended object
which it feels, or intuits, and so forth,
consciousness already contains in itself as consciousness
the determination of being-for-itself;
that is, it has in it the content of that object,
which is thus an idealization;
even as it intuits, or in general becomes
involved in the negative of itself, in the other,
it abides with itself.
Being-for-itself is the polemical,
negative relating to the limiting other
and, through this negation of the other,
is being-reflected-within-itself;
even though, side by side with this
immanent turning back of consciousness
and the ideality of its object,
the reality of this object is also retained,
for the object is at the same time
known as an external existence.
Consciousness is thus phenomenal,
or it is this dualism:
on the one side, it knows an external object
which is other than it;
on the other side, it is for-itself,
has this intended object in it as idealized,
abides not only by this other
but therein abides also with itself.
Self-consciousness, on the contrary, is
being-for-itself brought to completion and posited;
the side of reference to another,
to an external object, is removed.
Self-consciousness is thus the nearest
example of the presence of infinity;
granted, of a still abstract infinity,
but one which is of a totally different,
concrete determination than the
being-for-itself in general,
whose infinity still has only qualitative determinateness.

a. Existence and being-for-itself

As already mentioned, being-for-itself is
infinity that has sunk into simple being;
it is existence in so far as in the
now posited form of the immediacy
of being the negative nature of infinity,
which is the negation of negation,
is only as negation in general,
as infinite qualitative determinateness.
But in such a determinateness, wherein it is existence,
being is at once also distinguished
from this very being-for-itself
which is such only as infinite
qualitative determinateness;
nevertheless, existence is at the same time
a moment of being-for-itself,
for the latter certainly contains
being affected by negation.
So the determinateness which in existence as such is
an other, and a being-for-other,
is bent back into the infinite unity of being-for-itself,
and the moment of existence is present
in the being-for-itself as being-for-one.

b. Being-for-one

This moment gives expression to how the finite is
in its unity with the infinite or as an idealization.
Being-for-itself does not have negation in it as
a determinateness or limit,
and consequently also not as reference
to an existence other than it.
Although this moment is now being
designated as being-for-one,
there is yet nothing at hand for which it would be;
there is not the one of which it would be the moment.
There is in fact nothing of the sort
yet fixed in being-for-itself;
that for which something (and there is no something here)
would be, what the other side in general should be,
is likewise a moment,
itself only being-for-one,
not yet a one.
What we have before us, therefore,
is still an undistinguishedness of two sides
that may suggest themselves in the being-for-one;
there is only one being-for-another,
and since this is only one being-for-another,
it is also only being-for-one;
there is only the one ideality,
of that for which or in which
there should be a determination as moment,
and of that which should be the moment in it.
Being-for-one and being-for-itself do not therefore
constitute two genuine determinacies,
each as against the other.
Inasmuch as the distinction is momentarily assumed
and we speak of a-being-for-itself,
it is this very being-for-itself,
as the sublated being of otherness,
that refers itself to itself as to the sublated other,
is therefore for-one;
in its other it refers itself only to itself.
An idealization is necessarily for-one,
but it is not for an other;
the one, for which it is, is only itself.
The “I,” therefore, spirit in general,
or God, are idealizations,
because they are infinite;
as existents which are for-themselves, however,
they are not ideationally different
from that which is for-one.
For if they were different,
they would be only immediate,
or, more precisely, they would only be
existence and a being-for-another;
for if the moment of being for-one did
not attach to them,
it is not they themselves
but an other that would be
that which is for them.
God is therefore for himself,
in so far he is himself
that which is for him.

Being-for-itself and being-for-one are not, therefore,
diverse significations of ideality
but essential, inseparable, moments of it.

c. The one

Being-for-itself is the simple unity of
itself and its moments, of the being-for-one.
There is only one determination present,
the self-reference itself of the sublating.
The moments of being-for-itself have sunk into
an indifferentiation which is immediacy or being,
but an immediacy that is based on
the negating posited as its determination.
Being-for-itself is thus an existent-for-itself,
and, since in this immediacy its inner meaning vanishes,
it is the totally abstract limit of itself: the one.

Attention may be drawn in advance
to the difficulties that lie ahead
in the exposition of the development of the one,
and to the source of these difficulties.
The moments that constitute the concept of
the one as being-for-itself
occur in it one outside the other;
they are
(1) negation in general;
(2) two negations that are, therefore,
(3) the same,
(4) absolutely opposed;
(5) self-reference, identity as such;
(6) negative reference which is nonetheless self-reference.
These moments occur here apart because
the form of immediacy, of being, enters into
the being-for-itself as existent-for-itself;
because of this immediacy, each moment is posited
as a determination existent on its own,
and yet they are just as inseparable.
Hence, of each determination the opposite must equally be said;
it is this contradiction that causes the difficulty
that goes with the abstract nature of the moments.

B. THE ONE AND THE MANY

The one is the simple reference
of being-for-itself to itself
in which its moments have fallen together;
in which, therefore, being-for-itself has the form
of immediacy and its moments,
therefore, are now there as existents.
As the self-reference of the negative,
the one is a determining;
and, as self-reference,
it is infinite self-determining.
However, because of the present immediacy,
these distinctions are no longer only moments
of one and the same self-determination
but are at the same time posited as existents.
The ideality of the being-for-itself as a totality
thus turns at first into reality;
a reality, moreover, of the most
fixed and abstract kind, as a one.
In the one, the being-for-itself is
the posited unity of being and existence,
as the absolute union of the reference to another
and the reference to itself;
but also the determinateness of being
then enters into opposition to the determination
of the infinite negation, to self-determination,
so that what the one is in itself,
it is that now only in it,
and the negative consequently is
an other distinct from it.
What shows itself to be present
as distinct from the one is
the one's own self-determining;
its unity with itself, as thus distinct from itself,
is demoted to reference, and, as negative unity,
it is negation of itself as other,
the excluding of the one
as an other from itself,
from the one.

a. The one within

Within it, the one just is;
this, its being, is not an existence,
not a determination as reference to an other,
not a constitution;
it is rather its having
negated this circle of categories.
The one is not capable, therefore,
of becoming any other;
it is unalterable.
It is indeterminate,
yet no longer like being;
its indeterminateness is
the determinateness of self-reference,
absolutely determined being;
posited in-itselfness.
As negation which, in accordance with its concept,
is self-referring, it has distinction in it:
it directs away from itself towards another,
but this direction is immediately reversed,
because, according to this moment of self-determining,
there is no other to which it would be addressed,
and the directing reverts back to itself.
In this simple immediacy,
even the mediation of existence and ideality,
and with it all diversity and manifoldness,
have vanished.
In the one there is nothing;
this nothing, the abstraction of self-reference,
is here distinguished from the in-itselfness of the one;
it is a posited nothing, for this in-itselfness
no longer has the simplicity of the something,
but, as mediation, has rather the determination of being concrete;
taken in abstraction, it is indeed identical with the one,
but different from its determination.
So this nothing, posited as in the one,
is the nothing as the void.
The void is thus the quality of
the one in its immediacy.

b. The one and the void

The one is the void as the abstract self-reference of negation.
But the void, as nothing, is absolutely diverse
from the simple immediacy of the one,
from the being of the latter which is also affirmative,
and because the two stand in one single reference,
namely to the one, their diversity is posited;
however, as distinct from the affirmative being,
the nothing stands as void outside the one as existent.
Being-for-itself, determined in this way
as the one and the void,
has again acquired an existence.
The one and the void have their negative self-reference
as their common and simple terrain.
The moments of being-for-itself
come out of this unity,
become external to themselves;
for through the simple unity of the moments
the determination of being comes into play,
and the unity itself thus withdraws to one side,
is therefore lowered to existence,
and there it is confronted by its other determination
standing over against it, negation as such
and likewise as the existence of the nothing,
as the void.

c. Many ones

Repulsion

The one and the void constitute the first existence of being-for-itself.
Each of these moments has negation for its determination,
and is posited at the same time as an existence.
In accordance with this determination,
the one and the void are each the reference
of negation to negation as of an other to its other:
the one is negation in the determination of being; the void,
negation in the determination of non-being.
Essentially, however, the one
is only self-reference as referring negation,
that is, it is itself the same as the
void outside it is supposed to be.
Both are, however, also posited as
each an affirmative existence
(the one as being-for-itself as such,
the other as indeterminate existence in general)
and each as referring to the other as to an other existence.
Essentially, however, the being-for-itself of the one
is the ideality of the existence and of the other;
it does not refer to an other but only to itself.
But inasmuch as the being-for-itself is fixed as the one,
as existent for itself, as immediately present,
its negative reference to itself is
at the same time reference to an existent;
and since the reference is just as much negative,
that to which the being-for-itself refers remains
determined as an existence and as an other;
as essentially self-reference,
the other is not indeterminate negation like the void,
but is likewise a one.
The one is consequently a becoming of many ones.

Strictly speaking, however, this is not just a becoming;
for becoming is a transition of being into nothing;
the one, by contrast, becomes only a one.
The one, as referred to, contains
the negative as reference;
it has this reference, therefore, in it.
Hence, instead of a becoming,
the one's own immanent reference is,
first, present;
and, second, since this reference is negative
and the one is at the same time an existent,
the one repels itself from itself.
This negative reference of
the one to itself is repulsion.

This repulsion, as thus the positing of many ones
but through the one itself, is the one's
own coming-forth-from-itself,
but to such outside it as are themselves only ones.
This is repulsion according to the concept,
as it exists implicitly in itself.
The second repulsion is distinguished from it.
It is the one that first occurs to
the representation of external reflection,
not as the generation of ones
but only as the mutual holding off of ones
which are presupposed as already there.
To be seen now is how the first repulsion
that exists in itself determines itself
as the second, the external repulsion.

We must first establish the determinations
that the many ones have as such.
The becoming of the many, or their being produced,
immediately vanishes as the product of a positing;
what is produced are the ones, not for another,
but as infinitely referring to themselves.
The one repels only itself from itself;
it does not come to be but it already is;
that which is represented as the repelled is
equally a one, an existent;
repelling and being repelled applies
in like manner to both, and makes no difference.

The ones are thus presupposed
with respect to each other posited through
the repulsion of the one from itself;
presupposed, posited as non-posited;
their being-posited is sublated,
they are existents with respect to each other,
such as refer only to themselves.

Thus plurality appears not as an otherness,
but as a determination completely external to the one.
The one, in repelling itself, remains reference to itself,
just like that which is taken as repelled at the start.
That the ones are other to one another,
that they are brought together in
the determinateness of plurality,
does not therefore concern the one.
If the plurality were a
reference of the ones to one another,
the ones would then limit each other
and would have the being-for-other affirmatively in them.
Their connecting reference
(and this they have through their unity which is in itself),
as posited here, is determined as none;
it is again the previously posited void.
This void is their limit,
but an external limit in which
they are not supposed to be for one another.
The limit is that in which
the limited are just as much as are not;
but the void is determined as pure non-being,
and this alone constitutes the limit of the ones.

The repulsion of the one from itself is
the making explicit of what
the one is implicitly in itself;
but, thus laid out as one-outside-the-other,
infinity is here an infinity that has externalized itself,
and this it has done through the immediacy
of the infinite, of the one.
Infinity is just as much
the simple reference of the one to the one
as, on the contrary, the one's absolute lack of reference;
it is the former according to the simple affirmative reference
of the one to itself;
it is the latter according to the same reference as negative.
Or again, the plurality of the ones is
the one's own positing of the one;
the one is nothing but
the negative reference of the one to itself,
and this reference
[hence the one itself]
is the plural one.
But equally, plurality is utterly external to the one,
for the one is precisely the sublating of otherness;
repulsion is its self-reference and simple equality with itself.
The plurality of the ones is infinity as a contradiction
that unconstrainedly produces itself.

C. REPULSION AND ATTRACTION

a. Exclusion of the one

The many ones are each a being;
their existence or their reference to one another
is a non-reference, it is external to them:
the abstract void.
But they themselves are now this negative reference
to themselves as to existent others:
the demonstrated contradiction, the infinity posited
in the immediacy of being.
With this, repulsion now finds immediately before it
that which is repelled by it.
In this determination, it is an excluding;
the one repels from itself only the many not generated by it,
the ones not posited by it.
This repelling is mutual or from all sides, relative,
limited by the being of the ones.

Plurality is not at first posited otherness;
limit is only the void, only that in which the ones are not.
But in the limit they also are;
they are in the void,
or their repulsion is their
common connecting reference.

This mutual repulsion is the posited
existence of the many ones;
it is not their being-for-itself,
in accordance with which they would be
distinguished as many only in a third,
but is rather their own distinguishing
which preserves them.
They mutually negate themselves,
posit one another as being only for-one.
But at the same time they negate this
being only for-one just as much;
they repel the ideality that they have and are.
So the moments which in ideality are
absolutely united come apart.
In its being-for-itself, the one is also for-one;
but this one, for which it is, is itself;
its distinguishing from itself is immediately sublated.
But in the plurality the distinguished one has a being;
the being-for-one as has been determined in exclusion
is therefore a being-for-other.
Each thus comes to be repelled by an other,
is sublated and made into a one which is not for itself
but for-one, and an other one at that.

The being-for-itself of the many ones
thus shows itself to be their self-preservation
through the mediation of their mutual repulsion
in which they sublated themselves reciprocally
and posit the others as mere being-for-another.
But the self-preservation consists at the same time
in repelling this ideality and positing the ones
as not being for-an-other.
This self-preservation of the ones
through their negative reference to one another is,
however, rather their dissolution.

The ones not only are but maintain themselves
through their reciprocal exclusion.
First, it is in their being,
and indeed their being-in-itself as
contrasted with their reference to the other,
that they should now have a firm point of support
for their diversity as against their being negated;
this in-itselfness rests on their being ones.
But they all are this;
in their being-in-itself,
instead of having there their
firm point of support for their diversity,
they are all the same.
Second, their existence and their
way of relating to one another,
that is, their positing themselves as one,
is their reciprocal negating;
this, however, is likewise one and the same determination of
all through which they therefore posit themselves as identical;
just as, by being in themselves the same,
the ideality that should be posited in them
through others is their own,
and they thus repel just as little.
According to their being and positing,
they are, consequently, only one affirmative unity.

This consideration regarding the ones:
that from either side of their determination,
whether they just are or refer to one another,
they show themselves to be only one and the same,
indistinguishable, is a comparison that belongs to us.
Also to be seen, therefore, is what is posited in them
in their mutual reference itself.
They are (this much is presupposed in this reference)
and they are only inasmuch as
they negate themselves reciprocally
and at the same time keep away this ideality,
their being negated, from themselves, that is,
they negate the reciprocal negating.
But they are only inasmuch as they negate,
and so, since their reciprocal negating is negated,
their being is negated.
To be sure, since they are,
nothing would be negated through this negating
which for them is only something external;
this negating of the other rebounds off them,
coming their way only by striking their surface.
And yet, they turn back upon themselves
only by negating the others;
they are only as this mediation,
this turning back of theirs is their self-preservation
and their being-for- itself.
Since their negating is ineffectual
because of the resistance offered by the others,
whether as existents or as negating,
they do not return back to themselves,
do not preserve themselves, and so are not.
It was previously remarked that
the ones themselves are each a one like any other.
This is not just a matter of our
connecting them by way of reference,
of bringing them together externally;
repulsion is itself a referring;
the one that excludes the ones refers itself to them,
to the ones, that is, to itself.
The negative relating of the ones to one another is consequently
only a coming-together-with-oneself.
This identity in which their repelling crosses over is
the sublation of their diversity and externality
which they should have rather asserted with respect to
each other by excluding each other.

This self-positing-in-a-one of the many ones is attraction.

b. The one one of attraction

Repulsion is the fragmentation of the one,
first into the many of which
it is the negative relating,
since they presuppose each other as each existent;
it is only the ought of ideality;
this ideality will, however, be realized in attraction.
Repulsion passes over into attraction,
the many ones into one one.
Both, repulsion and attraction, are
at first distinguished from each other,
repulsion as the reality of the ones,
attraction as their posited ideality.
Attraction refers to repulsion
by having it for a presupposition.
Repulsion delivers the material for attraction.
If there were no ones,
there would be nothing to attract;
the representation of continuing attraction,
of the consumption of the ones,
presupposes an equally continuing generation of the ones;
the sense representation of spatial attraction
gives continuity to the flow of ones to be attracted;
to replace the atoms that vanish at the point of attraction,
another multitude comes forth from the void,
infinitely if one so wishes.
If attraction were represented as accomplished, that is,
the many as brought to the point of the one one,
the result would be just an inert one, no longer any attraction.
The ideality immediately present in attraction
still also has in it the determination of the negation of itself,
the many ones to which it refers;
attraction is inseparable from repulsion.

To attract pertains at first in equal measure
to each of the many ones as immediately present;
none has advantage over an other;
what would result then is an equilibrium in the attraction,
or more precisely, an equilibrium in the attraction
and the repulsion themselves, and an inert state of rest
without any ideality present there.
But there can be no question here of
any such immediately present one taking precedence over another,
for this would presuppose a determinate distinction between them;
attraction is rather the positing of
the given lack of distinction among the ones.
Attraction is itself the positing in the first place
of a one distinct from other ones;
these are only the immediate ones
that are to preserve themselves through repulsion;
through their posited negation, however, what proceeds
is the one of attraction
which is therefore determined as the mediated one,
the one posited as one.
The first ones, as immediate, do not in their ideality
return into themselves,
but have this ideality each in another.

The one one is, however,
ideality that has been realized,
posited in the one;
it attracts through the mediation of repulsion;
it contains in itself this mediation as its determination.
It thus does not swallow the attracted ones
within it as into one point, that is,
does not sublate them abstractly.
Since it contains repulsion in its determination,
the latter equally preserves the ones as many within it;
by its attracting, it musters, so to speak,
something before it, gains an area or a filling.
Thus there is in it the unity
of repulsion and attraction in general.

c. The connection of repulsion and attraction

The difference of the one and the many
has determined itself as a difference
of their mutual reference connecting them
which breaks down into two, repulsion and attraction,
each of which stands at first
outside the other on its own,
in such a way that the two are
essentially joined together nevertheless.
Their still indeterminate unity
must be brought out in greater detail.

As the fundamental determination of the one,
repulsion appears first, and it appears as immediate,
like its ones which are indeed generated by it
and yet are at the same time posited as immediate,
and it is therefore indifferent to the attraction
which is added to it externally as thus presupposed.
Rather, attraction is not presupposed by repulsion:
it is not supposed to have any part in the positing
and in the being of the latter, that is,
as if repulsion were not, already in it,
the negation of itself,
or the ones were not already negated in it.
In this way, we have repulsion in abstraction, by itself,
and attraction likewise holds out to the ones,
as each an existent, the side of an immediate existence
which comes to them by itself as an other.

If we take mere repulsion in this way, for itself,
it is then the dispersion of the many ones in indeterminacy,
outside the sphere of repulsion itself;
for repulsion is the negating of the connection
of the many to one another;
lack of connection is their determination when abstractly taken.
But repulsion is not just the void;
the ones, although unconnected,
do not repel what constitutes their determination,
do not exclude it.
Although negative, repulsion is nonetheless essentially connection;
the mutual repulsion and flight is not a liberation
from what is repelled and fled from;
that which is excluded still stands in connection
with what is excluded from it.
But this moment of connection is attraction,
which is thus within repulsion itself;
it is the negating of that abstract repulsion
by which the ones would each be an existent
referring only to itself without mutual exclusion.

But in starting with the repulsion
of the ones as immediately present there,
and with attraction consequently also posited
as intruding on them externally,
the two, repulsion and attraction,
are held apart as diverse
determinations despite their inseparability.
But it has been established that it is not
just repulsion which is presupposed by attraction,
but that there equally is present also
a reverse connection of repulsion to attraction,
and that repulsion no less has
attraction for its presupposition.

As thus determined, they are inseparable,
and at the same time each is determined
as an ought and a limitation with respect to the other.
Their ought is their abstract determinateness
as each an existent in itself,
a determinateness, however, which is thereby
directed beyond itself and refers to the other.
And so, through the mediation of the other, each is as other;
their self-subsistence consists in their being mutually posited
in this mediation as an other determining.
Thus, repulsion is the positing of the many;
attraction the positing of the one;
this latter is equally the negation of the many
and the former the negation of the ideality
of such a many in the one;
so that attraction too is attraction only through
the mediation of repulsion,
just as repulsion is repulsion
through the mediation of attraction.
In all this, however, the mediation of each with itself
through the other is in fact negated;
each of the two determinations is its own self-mediation.
This will result from a closer examination of the two determinations
and will bring us back to the unity of their concept.

In the first place, that each presupposes itself,
that in its presupposition each refers only to itself,
this is already present in the way the still relative
repulsion and attraction behave at first.

Relative repulsion is the mutual repulsion
of many ones which are already at hand,
supposedly immediately given.
But that there be many ones,
this is repulsion itself;
any presupposition that it would have is
only its own positing.
Moreover, the determination of the being
that would accrue to the ones
apart from their being posited
(whereby they would already be)
belongs likewise to repulsion.
Repelling is that through which
the ones manifest themselves
and maintain themselves as ones;
through which they are as such.
Their being is their repulsion itself,
which is thus not some relative
existence against another other
but relates itself throughout only to itself.

Attraction is the positing of the one as such,
of the real one, with respect to which
the existence of the many is determined
as only a vanishing idealization.
Attraction thus directly presupposes itself;
it presupposes itself in the determination namely,
of the many ones to be an idealization,
the same ones which are otherwise supposed
to have existence for themselves and to repel others,
including therefore any other that attracts.
Against this determination of repulsion,
the ones do not attain ideality only through
the relation to attraction;
on the contrary, the ideality is presupposed:
it is the ideality of the ones
as an existent in itself,
inasmuch as they, as ones
(including the one conceived as attracting),
are not distinguished from one another
but are one and the same.

This self-presupposing of the two determinations, each for itself,
implies further that each contains within itself the other as moment.
Self-presupposing in general is the positing of oneself
in a one as the negative of oneself (repulsion),
and what is presupposed in this positing is
the same as that which presupposes (attraction).
That each is in itself only a moment,
this is the transition of each from itself into the other,
the negation of itself in the other
and the positing of itself as the other of itself.
The one, as such, is thus a coming-out-of-itself;
is itself only the positing of itself as its other, as the many.
And the many, for its part, is
only the falling back upon itself
and the positing of itself as its other, as a one,
and is in this equally only the connecting of itself to itself,
each continuing itself in its other.
Therefore, the coming-out-of-itself (repulsion)
and the self-positing-as-one (attraction)
are already inherently present as undivided.
But in the repulsion and attraction which are relative, that is,
which presuppose immediate, determinedly existent ones,
it is posited that the two are each, within it, this negation of itself,
and consequently also the continuity of itself in its other.
The repulsion of the determinedly existent ones is
the self-preservation of the one
through the mutual holding off of the others,
so that (1) the other ones are negated in it
(this is the side of its existence or of its being-for-another
and is therefore attraction as the ideality of the ones);
and (2) the one is in itself,
without reference to the others
(however, not only has the in-itself in general
long since passed over into being-for-itself;
the one in itself, according to its determination,
is the coming to be of many).
The attraction of the existent ones is their ideality
and the positing of the one, and in this,
as both the negating and the producing of the one,
attraction sublates itself,
and as a positing within it of the one,
is the negative of itself:
it is repulsion.

With this, the development of being-for-itself
is completed and has attained its result.
In connecting itself to itself infinitely,
that is, as the posited negation of negation,
the one is the mediation by which it repels itself
as its absolute (that is, abstract) otherness (the many) from itself,
and in thus negatively connecting itself to this, its non-being,
it sublates it and is in it precisely only the connection to itself.
The one is only this becoming in which
the determination “it begins,” that is,
its being posited as an immediate existent,
and equally that, as result,
it has restored itself as the one,
that is, the equally immediate and exclusive one, have vanished;
the process which it is, posits and contains it
from all sides only as something sublated.
The sublation, determined at first only as a relative sublating of
the connection to another existent,
a connection which is therefore itself not
an indifferent repulsion and attraction,
equally proves itself to pass over into
the infinite connection of mediation
through the negation of the external
connection of immediate and determinate existents,
and to have for result precisely that becoming
which, in the instability of its moments, is the collapse,
or rather the going-together-with-itself, into simple immediacy.
This being, according to the determination
which it has now acquired, is quantity.

If we briefly review the moments
of this transition of quality into quantity,
we find that the qualitative has being and immediacy
for its fundamental determination,
and the limit and the determinateness are
in this immediacy so identical with the being of something,
that the something itself vanishes
along with its alteration;
as thus posited, it is determined as finite.
Because of the immediacy of this unity
in which the distinction has disappeared,
although it is implicitly present in
the unity of being and nothing,
the distinction falls outside
that unity as otherness in general.
This reference to the other contradicts the immediacy
in which qualitative determinateness is self-reference.
This otherness is sublated in
the infinity of the being-for-itself,
the being-for-itself that has realized
the distinction implicitly present in it
in the negation of negation:
has realized it as the one and the many
and as their connecting references,
and has also elevated the qualitative to true unity,
that is, a unity which is no longer immediate
but posited as accordant with itself.

This unity is, therefore,
(a) being, only as affirmative, that is,
immediacy self-mediated through the negation of negation:
being is posited as a unity permeating
its determinacies, limits, etc.,
which are posited in it as sublated;
(b) existence:
in this determination it is negation
or determinateness as moment of
the affirmative being;
yet this determinateness is no longer
immediate but reflected into itself,
refers not to another but to itself;
(c) absolutely-determined-being,
(d) absolute in-itselfness,
(e) the one;
(f) otherness as such is itself being-for-itself;
(g) being-for-itself:
as that being which persists
across the determinateness
and in which the one
and even the being-determined-in-itself
are posited as sublated.
The one is simultaneously determined
as having gone beyond itself and as unity;
the one, the absolutely determined limit,
is consequently posited as a limit which is none,
a limit which is in being but is indifferent to it.
